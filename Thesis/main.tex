\documentclass{article}
\usepackage{amsmath}
\usepackage{float}
\usepackage{algpseudocode,algorithm}

% Declaracoes em Português
\algrenewcommand\algorithmicend{\textbf{fim}}
\algrenewcommand\algorithmicdo{\textbf{faça}}
\algrenewcommand\algorithmicwhile{\textbf{enquanto}}
\algrenewcommand\algorithmicfor{\textbf{para}}
\algrenewcommand\algorithmicif{\textbf{se}}
\algrenewcommand\algorithmicthen{\textbf{então}}
\algrenewcommand\algorithmicelse{\textbf{senão}}
\algrenewcommand\algorithmicreturn{\textbf{devolve}}
\algrenewcommand\algorithmicfunction{\textbf{Função}}

% Rearranja os finais de cada estrutura
\algrenewtext{EndWhile}{\algorithmicend\ \algorithmicwhile}
\algrenewtext{EndFor}{\algorithmicend\ \algorithmicfor}
\algrenewtext{EndIf}{\algorithmicend\ \algorithmicif}
\algrenewtext{EndFunction}{\algorithmicend\ \textbf{função}}

% O comando For, a seguir, retorna 'para #1 -- #2 até #3 faça'
\algnewcommand\algorithmicto{\textbf{até}}
\algrenewtext{For}[3]%
{\algorithmicfor\ #1 $\gets$ #2 \algorithmicto\ #3 \algorithmicdo}

\begin{document}

\title{Uso de tamanhos variáveis de minimizers na busca inexata em padrões}
\author{Arthur Latache}

\maketitle
\begin{abstract}
Atualmente, na leitura do DNA nos precisamos de heurísticas, ja que o matching é inexato. A heurística mais utilizada é encontrar uma substring que representa a janela(Essas substrings são chamadas de minimizers), e procurar essa substring na sequência base e depois avaliar a ocorrência com um algoritmo mais custoso. Para encontrar o minimizer de uma janela precisamos de dois parametros: Tamanho da janela e do minimizer. Os algoritmos atuais usam parametros fixos para encontrar o minimizer. Nesse artigo exploramos a possibilidade e as vantagens de flexibilizar essas variaveis, com a tentativa de melhorar a busca de ocorrências na sequência base. Para avaliar os resultados compararemos com algumas das estrategias atuais para encontrar essas ocorrências.
\end{abstract}
\section{Introdução}
\subsection{Leitura do DNA atual}
\paragraph{}{Para ler o DNA nos usamos tecnologias que conseguem ler em pequentas partes(entre 200 e 20000 pares de bases). Precisamos posicionar essas pequenas leituras no genoma inicial para poder analizar essas leituras. Uma das heurísticas mais comuns usada para posicionar essas pequenas leituras se chama minimizers,e constitui de encontrar substrings que representam janelas maiores, para fazer encontrar essas substrings na sequência base.}
\subsection{Parametros da estrategia de minimizers}
\paragraph{}{A estrategia de minimizers geralmente utiliza dois parametros para encontrar uma ocorrência. Esses paramentros são tamanho da janela e do minimizer. O tamanho da janela especifica quantas bases vão ser definidas por um unico minimizer, esse tamanho é estritamente maior que o tamanho do minimizer, uma vez que o minimizer tem que estar contido na janela. As implementaçōes atuais usam parametros fixos, devido as restriçōes dos algoritmos usados, por isso exploramos a possibilidade de variar o tamanho do minimizer. Os algortimos desenvolvidos nesse artigo conseguem flexibilizar tanto to tamanho do minimizer como o tamanho da janela, com pouco ou nenhum custo extra.}
\section{Metodologia}
\subsection{Consideraçōes sobre a complexidade}
\paragraph{}{Para testar a possibilidade de flexibilizar o tamanho do minimizer, precisamos de novos algoritmos que quando variassemos os parametros não adicionam muito a complexidade em tempo da busca. O novo algoritmo de indexação devia ser no máximo \(O(N * logN)\), com \(N\) sendo o tamanho da sequência base.} 
\subsection{Definiçōes}
\paragraph{}{Aqui definimos propriedades e funçōes para ser usadas depois:}

\(Prop_1(a, b) = \forall z | z \leq min(|a|,|b|) (a < b \implies Substr(a, z) \leq Substr(a, z) )\)

\subsection{Preprocessamento}
\paragraph{}{Sendo \(Suf(i)\) um sufixo da sequência base, começando em \(i\) e \(Seq\) a sequência base, a seguinte propriedade é necessaria para continuar: \(\forall{i, j} | 0 <= i, j < |Seq|. Prop_1(Suf(i), Suf(j)) \)}
\subsubsection{Comparaçōes}
\paragraph{}{Nos dizemos que o menor entre dois sufixos é o menor lexicograficamente. Fazer essa comparação do jeito naive seria \(O(N)\), então usamos array de sufixos para otimizar esse processo. Contruimos um array de sufixos da string inicial e pegamos o inverso do array de sufixos: \(invSuf(pos) = y | suf\_array[y] = pos\), sendo \(suf\_array\) o array de sufixos. A construção do array de sufixos so ocorre uma vez e pode ser feita em \(O(N)\) sendo \(N\) o tamanho da sequência base. Com isso conseguimos fazer \(invSuf\) ser possivel em \(O(1)\).} 
\subsubsection{Preprocessamento da obtenção dos minimizers}
\paragraph{}{Usamos o algoritmo 1 para criar a QueryBase em \(O(N * log(N)\), que é usado para fazer as queries do minimizer em \(O(Log(N))\). Criamos uma Sparse Table com as posições iniciais dos menores sufixos de cada janela.}
\begin{algorithm}[H]
  \caption{Criar a QueryBase}
  \begin{algorithmic}[1]
    \Function{CreateMinimizerQueryBase}{invSuf, len}
      \State $minimizer\_positions := int[log_2(len)][len]$
      \For {$i$}{$0$}{$len - 1$}
        \State $minimizer\_positions[0][i] = i$
      \EndFor
      \For {$j$}{$1$}{$log_2(len)$}
        \For {$i$}{$0$}{$len - 2^j$}
          \State  $i1 := minimizer\_positions[j - 1][i]$
          \State  $i2 := minimizer\_positions[j - 1][i + 2^j]$
          \If {$invSuf(i1) > invSuf(i2)$}
            \State $minimizer\_positions[j - 1][i] = i2$
          \Else
            \State $minimizer\_positions[j - 1][i] = i1$
          \EndIf
        \EndFor
      \EndFor
    \EndFunction
  \end{algorithmic}
\end{algorithm} 
\subsubsection{Obtenção dos minimizers a partir da QueryBase}
\paragraph{}{Seguindo a premissa em \(2.3\), podemos achar a posição do menor sufixo dentro da janela \((initial\_position, initial\_position + window\_size - 1)\). Com isso, conseguimos encontrar o minimizer em \(O(1)\) a partir dessa posição.}
\begin{algorithm}[H]
  \caption{Obter a posição do minimizer numa janela}
  \begin{algorithmic}[1]
    \Function{GetMinimizerStartPos}{minimizer\_positions, initial\_pos, window\_size, minimizer\_size, invSuf}
      \State $lg := log_2(window\_size)$
      \State $i1 := minimizer\_positions[lg][initial\_pos]$
      \State $i2\_pos := initial\_pos + window\_size - minimizer\_size +1 - 2^{lg}$
      \State $i2 := minimizer\_positions[lg][i2\_pos]$
      \If {$invSuf(i1) < invSuf(i2)$}
        \State \Return $i1$
      \Else
        \State \Return $i2$
      \EndIf
    \EndFunction
  \end{algorithmic}
\end{algorithm} 
\subsection{Indexação}
\paragraph{}{Para fazer a busca criamos um indice que usa como chave o minimizer, e contem todas as posições da sequência base onde ele ocorre. Quando usamos o tamanho do minimizer variável, salvamos isso para todos os tamanhos de minimizers que irão ser usados. Nessa pesquisa os tamanhos que utilizamos foram \([16, 32, 64, 128]\). O indice usa um tamanho de janela fixo, nesse caso usamos 200 para fazer os testes.}
\begin{algorithm}[H]
  \caption{Obter a posição do minimizer numa janela}
  \begin{algorithmic}[1]
    \Function{CreateIndex}{minimizer\_positions, len, window\_size, minimizer\_sizes, invSuf}
      \State $index := \{\}$
      \For {$j$}{$0$}{$|minimizer\_sizes| - 1$}
        \State $minimizer\_size := minimizer\_sizes[j]$
        \For {$i$}{$0$}{$len - 1$}
          \State $GMSP := GetMinimizerStartPos$
          \State $msizes := minimizer\_sizes$
          \State $msize := minimizer\_size$
          \State $min\_pos := GMSP(msizes, i, window\_size, msize, invSuf)$
          \State $minimizer := Substr(min\_pos, min\_pos + msize)$
          \If {$minimizer not in index$}
            \State $index[minimizer] := {}$
          \EndIf
          \State $index[minimizer] = index[minimizer] \cup \{min\_pos\}$
        \EndFor
      \EndFor
      \State \Return index
    \EndFunction
  \end{algorithmic}
\end{algorithm} 
\subsection{Criando fingerprint de reads}
\paragraph{}{Criamos uma fingerprint(MiniMap) de cada read, sumarizando os minimizers e onde eles ocorrem. Uma fingerprint é um conjunto de tuplas compostas por onde a read devia ter ocorrido na sequência base, a posição do minimizer e o tamanho do minimizer, para cada posição da janela.  }
\subsubsection{Fingerprint com minimizer de tamanho fixo}
\subsubsection{Fingerprint com minimizer de tamanho variável}
\subsection{Posicionamento das reads}
\section{Resultados}
\paragraph{}{Para avaliar se esse resultado melhora as buscas, de}
\section{Discussão}
Write your conclusion here.
\end{document}
