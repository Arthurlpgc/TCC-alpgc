%% Template para dissertação/tese na classe UFPEthesis
%% versão 0.9.2
%% (c) 2005 Paulo G. S. Fonseca
%% www.cin.ufpe.br/~paguso/ufpethesis

%% Carrega a classe ufpethesis
%% Opções: * Idiomas
%%           pt   - português (padrão)
%%           en   - inglês
%%         * Tipo do Texto
%%           bsc  - para monografias de graduação
%%           msc  - para dissertações de mestrado (padrão)
%%           qual - exame de qualificação doutorado
%%           prop - proposta de tese doutorado
%%           phd  - para teses de doutorado
%%         * Mídia
%%           scr  - para versão eletrônica (PDF) / consulte o guia do usuario
%%         * Estilo
%%           classic - estilo original à la TAOCP (deprecated)
%%           std     - novo estilo à la CUP (padrão)
%%         * Paginação
%%           oneside - para impressão em face única
%%           twoside - para impressão em frente e verso (padrão)
\documentclass[bsc, oneside, scr, 12pt]{ufpethesis}

\usepackage{setspace}
\usepackage{lineno}

\newenvironment{comment}{\itshape\noindent<begin comment -- ommit this in the final version>\par}{\par\noindent<end comment>\rm}

\newcommand{\minimiser}{\textit{minimiser}\/\xspace}
\newcommand{\minimisers}{\textit{minimisers}\/\xspace}


%% Preâmbulo:
%% coloque aqui o seu preâmbulo LaTeX, i.e., declaração de pacotes,
%% (re)definições de macros, medidas, etc.

%% Identificação:

% Universidade
% e.g. \university{Universidade de Campinas}
% Na UFPE, comente a linha a seguir
%\university{<NOME DA UNIVERSIDADE>}

% Endereço (cidade)
% e.g. \address{Campinas}
% Na UFPE, comente a linha a seguir
%\address{<CIDADE DA IES>}

% Instituto ou Centro Acadêmico
% e.g. \institute{Centro de Ciências Exatas e da Natureza}
% Comente se não se aplicar
\institute{Centro de Informática}

% Departamento acadêmico
% e.g. \department{Departamento de Informática}
% Comente se não se aplicar
%\department{<NOME DO DEPARTAMENTO>}

% Programa de pós-graduação
% e.g. \program{Pós-graduação em Ciência da Computação}
\program{Graduação em Ciência da Computação}

% Área de titulação
% e.g. \majorfield{Ciência da Computação}
\majorfield{Ciência da Computação}

% Título da dissertação/tese
% e.g. \title{Sobre a conjectura $P=NP$}
\title{Casamento aproximado de padrões baseado em índices de \textit{minimisers} de comprimento variável}

% Data da defesa
% e.g. \date{19 de fevereiro de 2003}
\date{04 de Julho de 2019}

% Autor
% e.g. \author{José da Silva}
\author{Arthur Latache Pimentel Gesteira Costa}

% Orientador(a)
% Opção: [f] - para orientador do sexo feminino
% e.g. \adviser[f]{Profa. Dra. Maria Santos}
\adviser{Paulo Gustavo Soares da Fonseca}

% Orientador(a)
% Opção: [f] - para orientador do sexo feminino
% e.g. \coadviser{Prof. Dr. Pedro Pedreira}
% Comente se não se aplicar
%\coadviser{NOME DO(DA) CO-ORIENTADOR(A)}

%% Inicio do documento
\begin{document}

%%
%% Parte pré-textual
%%
\frontmatter

% Folha de rosto
% Comente para ocultar
\frontpage

% Portada (apresentação)
% Comente para ocultar
\presentationpage

% Dedicatória
% Comente para ocultar
%\begin{dedicatory}
%<DIGITE A DEDICATÒRIA AQUI>
%\end{dedicatory}

% Agradecimentos
% Se preferir, crie um arquivo à parte e o inclua via \include{}
%\acknowledgements
%<DIGITE OS AGRADECIMENTOS AQUI>

% Epígrafe
% Comente para ocultar
% e.g.
%  \begin{epigraph}[Tarde, 1919]{Olavo Bilac}
%  Última flor do Lácio, inculta e bela,\\
%  És, a um tempo, esplendor e sepultura;\\
%  Ouro nativo, que, na ganga impura,\\
%  A bruta mina entre os cascalhos vela.
%  \end{epigraph}
%\begin{epigraph}[<NOTA>]{<AUTOR>}
%<DIGITE AQUI A CITAÇÂO>
%\end{epigraph}

\doublespace
\linenumbers


% Resumo em Português
% Se preferir, crie um arquivo à parte e o inclua via \include{}
\resumo
<DIGITE O RESUMO AQUI>
% Palavras-chave do resumo em Português
\begin{keywords}
<DIGITE AS PALAVRAS-CHAVE AQUI>
\end{keywords}

% Resumo em Inglês
% Se preferir, crie um arquivo à parte e o inclua via \include{}
\abstract
% Palavras-chave do resumo em Inglês
\begin{keywords}
<DIGITE AS PALAVRAS-CHAVE AQUI>
\end{keywords}

% Sumário
% Comente para ocultar
\tableofcontents

% Lista de figuras
% Comente para ocultar
\listoffigures

% Lista de tabelas
% Comente para ocultar
\listoftables



%%
%% Parte textual
%%
\mainmatter

% É aconselhável criar cada capítulo em um arquivo à parte, digamos
% "capitulo1.tex", "capitulo2.tex", ... "capituloN.tex" e depois
% incluí-los com:
% \include{capitulo1}
% \include{capitulo2}
% ...
% \include{capituloN}


\chapter{Introdução}

\begin{comment}
Neste capítulo devem ser respondidas as questões Q1-2. O capítulo deve conter um background da área necessário para introdução e contextualização do problema específico. Deve-se contar uma história que motive o trabalho, de forma que fique justificada sua importância no contexto da área. Uma vez que o tema esteja suficientemente motivado e justificado, o problema deve ser colocado de maneira clara, explicitando-se os objetivos gerais e específicos do trabalho. Em outras palavras, temos até aqui um desenvolvimento dos dois primeiros tópicos do resumo. O capítulo deve terminar com uma síntese dos resultados/contribuições alcançadas e com uma visão geral dos capítulos subsequentes.

Idealmente, o Cap. 1 deve ser compreensível  pelo menos por qualquer pessoa da área (e.g. de Computação), mesmo que não seja especialista  no tema.
\end{comment}

\section{Motivação}

Roteiro:
\begin{enumerate}
\item Problema do casamento aproximado de texto: definição informal  e suas aplicações. Fixar notação básica: padrão, texto, ...
\item Problema do casamento de texto e aplicações a Biologia Computacional: busca por regiões homólogas, resequenciamento.
\item Solução online versus indexada: descrição informal e diferenças em termos de eficiência.
\item Índices completos versus parciais.
\item Visão geral da heurística \textit{seed\&extend}
\end{enumerate}

\section{Objetivo}

A principal contribuição deste trabalho é propor um algoritmo para busca aproximada de padrões com auxílio de um índice parcial de subsequências chamadas \textit{minimisers}. O nosso algoritmo está inspirado em trabalhos anteriores, em particular nas ferramentas de alinhamento \textit{TAPyR} \cite{Fernandes2011} e  \textit{Minimap} \cite{Li2016}. Similar ao TAPyR, nosso algoritmo baseia-se no uso da estratégia \textit{seed\&extend} com uma heurística para combinação de sementes de comprimento variável. Entretanto, diferente do \textit{TAPyr} e a exemplo do \textit{Minimap}, nosso algoritmo baseia-se num índice parcial e mais eficiente de \minimisers. Nosso método diferencia-se do \textit{Minimap} por empregar \minimisers de tamanhos distintos para oferecer um melhor balanço entre a sensibilidade e especificidade do índice.


\section{Estrutura desta monografia}

O restante da presente monografia está estruturada da seguinte maneira.

No Capítulo 2, nós introduzimos os conceitos fundamentais e formalizamos o problema do casamento aproximado de padrões com índices parciais baseados em \minimisers. Nós apresentamos os trabalhos relacionados da literatura, em particular o algoritmo \textit{minimap}\cite{} no qual o nosso algoritmo é baseado.

No Capítulo 3, apresentamos uma descrição detalhada dos algoritmos propostos, salientando detalhes de implementação relevante, e apresentando uma análise assintótica teórica dos custos computacionais incorridos.

No Capítulo 4, são apresentados os resultados experimentais obtidos com a implementação de protótipos dos algoritmos propostos, com o objetivo de demonstrar as vantagens relativas das heurísticas propostas, principalmente quanto ao uso de \minimisers de comprimentos variáveis.

No Capítulo 5 nós apresentamos uma breve discussão acerca dos resultados obtidos, com conclusões gerais e indicações sobre desenvolvimentos futuros.




\chapter{Fundamentação Teórica}

\begin{comment}
	Este capítulo deve conter uma descrição técnica e precisa dos conceitos e problemas abordadeos e um levantamento bibliográfico crítico e circunstanciado do tema do trabalho, desde os trabalhos seminais até os desenvolvimentos recentes relacionados ao problema de pesquisa abordado. O objetivo do capítulo é demonstrar um conhecimento em largura - da área como um todo - e em profundidade - quanto ao problema específico - em nível de especialista. A narrativa deve ser técnica, precisa e fundamentada. As dificuldades, avanços e problemas em aberto relativas ao problema devem ser colocados de maneira razoavelmente detalhada, fazendo-se uso extenso de referências bibliográficas. Apesar disso, esta revisão da literatura deve "contar uma história" de forma coerente e não ser apenas uma coleção de referências. Deve-se procurar um fio condutor que leve ao problema específico abordado.
\end{comment}

Breve introdução.


\section{Casamento Aproximado de Texto}

Definição formal, distância de edição, notação...

\section{Índices de Texto}

\subsection{Índices completos}

Árvores e vetores de sufixos.

\subsection{Índices parciais}

Índices de kmers.

\section{Minimisers}

\section{Seed\&Extend}

\section{Casamento aproximado por seed\&extend com índices completos}

TAPyR

\section{Casamento aproximado por seed\&extend com índices pariciais de \minimisers}

Minimap



\chapter{Métodos}

\begin{comment}
	Este capítulo deve conter uma descrição precisa, com linguagem rigorosa e notação apropriada de "como" o você abordou o problema. Devem ser descritos os detalhes do seu método, com os fundamentos teóricos na forma de definições, proposições e suas provas, os algoritmos implementados, com seus pseudocódigos e complexidades teóricas. Também devem ser descritos aspectos práticos relevantes relativos à implementação da metodologia, quando aplicável. As decisões devem ser justificadas tecnicamente, à luz da literatura,  dos resultados conhecidos e da situação prática em questão.
\end{comment}

Breve introdução.

\section{Visão geral do método}

Apresentar uma visão geral do método com suas componentes principais (índice, algoritmo de busca). Apresentar um diagrama com as caixinhas da metodologia, similar às figuras 1 e 2 de https://www.cin.ufpe.br/~paguso/courses/if767/2017-2/docs/projeto2.pdf.

\section{Construção do Índice de \minimisers de tamanho variável}

Construção do o auxílio da suffix array da janela.
Descrição dos algoritmos e complexidade


\section{Casamento aproximado por seed\&extend}

Fingerprints, heurística de combinação das seeds,... descrição dos algoritmos e complexidade.



\chapter{Resultados}

\begin{comment}
	Este capítulo deve conter uma discussão detalhada sobre os resultados obtidos. Os experimentos devem ser descritos em detalhe, incluindo os objetivos ou hipóteses testadas, os dados e as métricas. Os resultados devem ser apresentados de maneira sistemática, com o auxílio de tabelas, diagramas, gráficos, etc., e analisados de maneira rigorosa com auxílio das técnicas estatísticas, conforme apropriado. Mais uma vez, o objetivo do capítulo é "contar a história" dos resultados obtidos. Assim, deve-se tomar cuidado de escolher os experimentos que suportam as conclusões alcançadas e apresentá-los de forma a facilitar a interpretação. A tabulação de dados brutos volumosos deve ser evitada em favor de apresentações mais sintéticas. Se estritamente necessário, os resultados completos podem ser disponibilizados nos apêndices ou como material suplementar. Respeitando o método científico, os resultados devem ser descritos em suficiente detalhe a ponto de serem reproduzidos ou auditados.
\end{comment}


Breve introdução.

\section{Implementação}

Comentar como foi feita a implementação do protótipo em particular
\begin{itemize}
\item  linguagem de programação
\item  bibliotecas e estruturas de dados principais que afetam o desempenho dos algoritmos
\item limitações conhecidas
\end{itemize}

Fornecer um apontador para onde o código pode ser obtido.


\section{Ambiente experimental}

Detalhar o ambiente experimental (máquina) e dados utilizados nos experimentos. Fornecer apontadores para que possam ser recuperados. 

\section{Análise de sensibilidade e especificidade de \minimisers de diferentes comprimentos}

Apresentar os resultados de sensibilidade/especificidade para os diferentes tamanhos de \minimisers de forma a motivar e justificar a flexibilização para  \minimisers de tamanho variável.


\section{Análise do casamento de padrões}

Apresentar os resultados do casamento de padrões com \minimisers de tamanho fixo comparado com a versão que combina os \minimisers de tamanho variado.


Nessas duas seções, o rito é o seguinte:

\begin{enumerate}
\item Explicar o experimento, incluindo a entrada (dados) e o que se pretendia medir.
\item Apresentar os dados dos resultados experimento: gráficos, tabelas, etc.
\item Discutir brevemente os resultados do experimento, explicando o por quê desses resultados quando apropriado. Os resultados confirmam as previsões teóricas? Por que?
\end{enumerate}


\chapter{Conclusões}

\begin{comment}
Este último capítulo deve conter uma discussão final do trabalho, eventualmente precedido por uma breve recapitulação dos principais pontos dos capítulos anteriores. O capítulo deve enfatizar as contribuições alcançadas pelo trabalho e uma análise crítica das suas limitações, apontando direções para desenvolvimentos futuros.
\end{comment}

Roteiro:

\begin{enumerate}
\item Recapitulação do que foi feito
\item Resumo dos resultados obtidos com análise crítica. De forma geral, os resultados esperados foram atingidos? 
\item Limitações. O que não correu tão bem? Por que?
\item Trabalhos futuros. O que mais poderia ser feito? Uma óbvia é que a implementação é um protótipo implementado com o objetivo de verificar a viabilidade do método, sendo necessária uma implementação mais eficiente pra análise do desempenho num cenário real.
\end{enumerate}

Terminar com uma nota positiva.




%%
%% Parte pós-textual
%%
\backmatter

% Apêndices
% Comente se não houver apêndices
\appendix

% É aconselhável criar cada apêndice em um arquivo à parte, digamos
% "apendice1.tex", "apendice.tex", ... "apendiceM.tex" e depois
% incluí-los com:
% \include{apendice1}
% \include{apendice2}
% ...
% \include{apendiceM}


% Bibliografia
% É aconselhável utilizar o BibTeX a partir de um arquivo, digamos "biblio.bib".
% Para ajuda na criação do arquivo .bib e utilização do BibTeX, recorra ao
% BibTeXpress em www.cin.ufpe.br/~paguso/bibtexpress
\nocite{*}
\bibliographystyle{plain}
\bibliography{monografia}

% Cólofon
% Inclui uma pequena nota com referência à UFPEThesis
% Comente para omitir
\colophon

%% Fim do documento
\end{document}
